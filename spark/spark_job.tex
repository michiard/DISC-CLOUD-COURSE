%%%%%%%%%%%%%%%%%%%%%%%%%%%%%%%%%%%%%%%%%%%%%%%%%%%%%%%%%%
\frame {\frametitle{Spark Components}
%%%%%%%%%%%%%%%%%%%%%%%%%%%%%%%%%%%%%%%%%%%%%%%%%%%%%%%%%%
\begin{figure}[h]
  \centering
  \includegraphics[scale=0.33]{./Figures/spark_components}
  \label{fig:spark_components}
\end{figure}
}

%%%%%%%%%%%%%%%%%%%%%%%%%%%%%%%%%%%%%%%%%%%%%%%%%%%%%%%%%%
% \frame {\frametitle{A Simple Job Example in Spark}
\begin{frame}[fragile]
\frametitle{A Simple Job Example in Spark}
%%%%%%%%%%%%%%%%%%%%%%%%%%%%%%%%%%%%%%%%%%%%%%%%%%%%%%%%%%


\begin{lstlisting}
val sc = new SparkContext("spark://...", "MyJob", home, jars) 

val file = sc.textFile("hdfs://...") // This is an RDD

val errors = file.filter(_.contains("ERROR")) // This is an RDD

errors.cache()

errors.count() // This is an action
\end{lstlisting}

\end{frame}


%%%%%%%%%%%%%%%%%%%%%%%%%%%%%%%%%%%%%%%%%%%%%%%%%%%%%%%%%%
\frame {\frametitle{The RDD graph: dataset vs. partition views}
%%%%%%%%%%%%%%%%%%%%%%%%%%%%%%%%%%%%%%%%%%%%%%%%%%%%%%%%%%
% missing: what is an rdd? difference between RDD and partition?

\begin{columns}[c]
	\column{.5\textwidth}
		
			\begin{figure}[h]
			  \centering
			  \includegraphics[scale=0.25]{./Figures/spark_rdd}
			  \label{fig:spark_components}
			\end{figure}
	
	\column{.5\textwidth}
		
			\begin{figure}[h]
			  \centering
			  \includegraphics[scale=0.25]{./Figures/spark_partition}
			  \label{fig:spark_components}
			\end{figure}

\end{columns}
}

%%%%%%%%%%%%%%%%%%%%%%%%%%%%%%%%%%%%%%%%%%%%%%%%%%%%%%%%%%
\frame {\frametitle{Data Locality}
%%%%%%%%%%%%%%%%%%%%%%%%%%%%%%%%%%%%%%%%%%%%%%%%%%%%%%%%%%
\begin{itemize}
	\item First run: data not in cache, so use HadoopRDD's locality prefs (from HDFS)
	\item Second run: FilteredRDD is in cache, so use its locations
	\item If something falls out of cache, go back to HDFS
\end{itemize}
}
