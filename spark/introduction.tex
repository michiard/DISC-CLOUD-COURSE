%%%%%%%%%%%%%%%%%%%%%%%%%%%%%%%%%%%%%%%%%%%%%%%%%%%%%%%%%%
\frame {\frametitle{What is Apache Spark}
%%%%%%%%%%%%%%%%%%%%%%%%%%%%%%%%%%%%%%%%%%%%%%%%%%%%%%%%%%
\begin{itemize}
  \item Project goals:
  \begin{itemize}
    \item Generality: diverse workloads, operators, job sizes
    \item Low latency: sub-second
    \item Fault tolerance: faults are the norm, not the exception
    \item Simplicity: often comes from generality
  \end{itemize}
\end{itemize}

}

%%%%%%%%%%%%%%%%%%%%%%%%%%%%%%%%%%%%%%%%%%%%%%%%%%%%%%%%%%
\frame {\frametitle{Motivations}
%%%%%%%%%%%%%%%%%%%%%%%%%%%%%%%%%%%%%%%%%%%%%%%%%%%%%%%%%%
\begin{itemize}
  \item Software engineering point of view
  \begin{itemize}
    \item Hadoop code base is huge
    \item Contributions/Extensions to Hadoop are cumbersome
    \item Java-only hinders wide adoption, but Java support is fundamental
  \end{itemize}

\vspace{20pt}

  \item System/Framework point of view
  \begin{itemize}
    \item Unified pipeline
    \item Simplified data flow
    \item Faster processing speed
  \end{itemize}

\vspace{20pt}

  \item Data abstraction point of view
  \begin{itemize}
    \item New fundamental abstraction RDD
    \item Easy to extend with new operators
    \item More descriptive computing model
  \end{itemize}
\end{itemize}
}

%%%%%%%%%%%%%%%%%%%%%%%%%%%%%%%%%%%%%%%%%%%%%%%%%%%%%%%%%%
\frame {\frametitle{Hadoop: No Unified Vision}
%%%%%%%%%%%%%%%%%%%%%%%%%%%%%%%%%%%%%%%%%%%%%%%%%%%%%%%%%%
\begin{figure}[h]
  \centering
  \includegraphics[scale=0.25]{./Figures/hadoop_modules}
  \label{fig:spark_modules}
\end{figure}
\begin{itemize}
    \item Sparse modules
    \item Diversity of APIs
    \item Higher operational costs
\end{itemize}
}

%%%%%%%%%%%%%%%%%%%%%%%%%%%%%%%%%%%%%%%%%%%%%%%%%%%%%%%%%%
\frame {\frametitle{SPARK: A Unified Pipeline}
%%%%%%%%%%%%%%%%%%%%%%%%%%%%%%%%%%%%%%%%%%%%%%%%%%%%%%%%%%
\begin{figure}[h]
  \centering
  \includegraphics[scale=0.4]{./Figures/spark_modules}
  \label{fig:spark_modules}
\end{figure}
\begin{itemize}
    \item Spark Streaming (stream processing)
    \item GraphX (graph processing)
    \item MLLib (machine learning library)
    \item Spark SQL (SQL on Spark) 
\end{itemize}
}

%%%%%%%%%%%%%%%%%%%%%%%%%%%%%%%%%%%%%%%%%%%%%%%%%%%%%%%%%%
\frame {\frametitle{A Simplified Data Flow}
%%%%%%%%%%%%%%%%%%%%%%%%%%%%%%%%%%%%%%%%%%%%%%%%%%%%%%%%%%
\begin{figure}[h]
  \centering
  \includegraphics[scale=0.25]{./Figures/data_flow_hadoop}
  \label{fig:data_flow_hadoop}
\end{figure}

\begin{figure}[h]
  \centering
  \includegraphics[scale=0.25]{./Figures/data_flow_spark}
  \label{fig:data_flow_spark}
\end{figure}
}

%%%%%%%%%%%%%%%%%%%%%%%%%%%%%%%%%%%%%%%%%%%%%%%%%%%%%%%%%%
\frame {\frametitle{Hadoop: Bloated Computing Model}
%%%%%%%%%%%%%%%%%%%%%%%%%%%%%%%%%%%%%%%%%%%%%%%%%%%%%%%%%%
\begin{figure}[h]
  \centering
  \includegraphics[scale=0.23]{./Figures/wc_hadoop}
  \label{fig:wc_hadoop}
\end{figure}
}

%%%%%%%%%%%%%%%%%%%%%%%%%%%%%%%%%%%%%%%%%%%%%%%%%%%%%%%%%%
\frame {\frametitle{SPARK: Descriptive Computing Model}
%%%%%%%%%%%%%%%%%%%%%%%%%%%%%%%%%%%%%%%%%%%%%%%%%%%%%%%%%%
\begin{figure}[h]
  \centering
  \includegraphics[scale=0.36]{./Figures/wc_spark}
  \label{fig:wc_spark}
\end{figure}

\begin{itemize}
  \item Organize computation into multiple stages in a processing pipeline
  \begin{itemize}
    \item \textbf{Transformations} apply user code to distributed data in parallel
   \item \textbf{Actions} assemble final output of an algorithm, from distributed data
  \end{itemize}
    
  \end{itemize}  
}

%%%%%%%%%%%%%%%%%%%%%%%%%%%%%%%%%%%%%%%%%%%%%%%%%%%%%%%%%%
\frame {\frametitle{Faster Processing Speed}
%%%%%%%%%%%%%%%%%%%%%%%%%%%%%%%%%%%%%%%%%%%%%%%%%%%%%%%%%%
\begin{itemize}
  \item Let's focus on iterative algorithms
  \begin{itemize}
    \item Spark is faster thanks to the simplified data flow
    \item We avoid materializing data on HDFS after each iteration
  \end{itemize}

\vspace{20pt}

  \item Example: k-means algorithm
  \begin{itemize}
    \item Step 1: Place randomly initial group centroids into the space
    \item Step 2: Assign each object to the group that has the closest centroid.
    \item Step 3: Recalculate the positions of the centroids.
    \item Step 4: If the positions of the centroids didn't change go to the next step, else go to Step 2.
    \item Step 5: End.
  \end{itemize}
\end{itemize}
}

%%%%%%%%%%%%%%%%%%%%%%%%%%%%%%%%%%%%%%%%%%%%%%%%%%%%%%%%%%
\frame {\frametitle{Code Base (2012)}
%%%%%%%%%%%%%%%%%%%%%%%%%%%%%%%%%%%%%%%%%%%%%%%%%%%%%%%%%%
\begin{figure}[h]
  \centering
  \includegraphics[scale=0.36]{./Figures/code_base}
  % \caption{Spark code base, as of 0.6.x (2012).}
  \label{fig:code_base}
\end{figure}

\begin{itemize}
  \item 2012 (version 0.6.x): 20,000 lines of code
  \item 2014 (branch-1.0): 50,000 lines of code
\end{itemize}
}
